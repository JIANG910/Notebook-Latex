\chapter{定积分}
\section{定积分的基本概念}
\subsection{曲边梯形的面积}
设$y=f(x)$在区间$[a,b]$上非负,连续。由直线$x=a,x=b,y=0$及曲线$y=f(x)$所围成的图形称为\highlight{red}{\index{QBTX@曲边梯形}曲边梯形},其中曲线弧称为\highlight{red}{曲边}。\\
\thispagestyle{empty}
\kg 下面用\highlight{red}{\index{YSF@元素法}元素法}\footnote{元素法在定积分的应用会详细讲解}详细地说明曲边梯形面积的求法。\\
\kg (1)\enspace\textbf{分割}。用任意一组分点把区间$[a,b]$分成长度为$\Delta x_i$($i=1,2,\cdots,n$)的$n$的小区间$[x_{i-1},x_i]$,相应地把曲边梯形分成$n$个窄曲边梯形,第$i$个窄曲边梯形的面积设为$\Delta S_i$,则
\begin{equation}
	S=\sum_{i=1}^{n}\Delta S_i
\end{equation}
\kg (2)\textbf{计算$\Delta S_i$的近似值}(利用窄曲边梯形的面积$\approx$窄边矩形的面积=高$\times$宽,在每个小区间$[x_{i-1},x_i]$上用其中某一点$\xi_i$处的高来近似代替同一个小区间上窄矩形的变高)
\begin{equation}
	\Delta S_i\approx f(\xi_i)\Delta x_i(x_{i-1}\leq \xi_i\leq x_i)
\end{equation}
\kg (3)\textbf{求和},得$S$的近似值
\begin{equation}
	S=\sum_{i=1}^{n}\Delta S_i\approx\sum_{i=1}^{n} f(\xi_i)\Delta x_i
\end{equation}
\kg (4)\textbf{求极限},记$\lambda=\text{max}\{\Delta x_i,1\leq i\leq n\}$,($\lambda$的几何意义是所有分得的小区间中长度最大的区间。当$\lambda\to0$时,所有的小区间长度都趋于0,这个时候分得的每个区间足够小,上式便不是估计式,而是等式)得
\begin{equation}
	S=\sum_{i=1}^{n}\Delta S_i=\lim\limits_{\lambda\to 0}\sum_{i=1}^{n}f(\xi_i)\Delta x_i
\end{equation}
\subsection{定积分的定义}
通过计算曲边梯形的面积,我们可以类比得到定积分的定义。
\\ 

\sj
\defination[定积分定义]
设函数$f(x)$在$[a,b]$上有界,在区间$[a,b]$中任意插入若干个分点
\begin{equation}
	\nonumber
	a=x_0<x_1<x_2<\cdots<x_{n-1}<x_n=b
\end{equation}
把区间$[a,b]$分成$n$个小区间:
\begin{equation}
	[x_1,x_2],[x_2,x_3],\cdots,[x_{n-1},x_n]
\end{equation}
各个小区间的长度依次为:
\begin{equation}
	\nonumber
	\Delta x_1=x_2-x_1,\Delta x_2-x_3,\cdots,\Delta x_n=x_n-x_{n-1}
\end{equation}
在每个小区间$[x_{i-1},x_i]$上任取一点$\xi_i(x_{i-1}\leq \xi_i\leq x_i)$,作函数值$f(\xi_i)$与小区间长度$\Delta x_i$的乘积$f(\xi_i)\Delta x_i$,并求和
\begin{equation}
	S=\sum_{i=1}^{n}f(\xi_i)\Delta x_i
\end{equation}
\kg 记$\lambda=\text{max}\{\Delta x_i,1\leq i\leq n\}$\footnote{无特殊说明,在本章节中$\lambda$都表示$\max \{\Delta x_i,1\leq x_i\leq n\}$},如果$\lambda\to 0$时,上式的极限总存在,且与闭区间$[a,b]$的分法及点$\xi_i$的取法无关,那么称这个极限为函数$f(x)$在区间$[a,b]$上的\highlight{red}{\index{DJF@定积分}定积分}(简称\highlight{red}{\index{JF@积分}积分}),记作
\begin{equation}
	\int_{a}^{b}f(x)\d x=\lim\limits_{\lambda\to0}\sum_{i=1}^{n}f(\xi_i)\Delta x_i
\end{equation}
\kg 其中函数$f(x)$叫做\highlight{red}{\index{BJHS@被积函数}被积函数},$f(x)\d x$叫做\highlight{red}{\index{BJBDS@被积表达式}被积表达式},$x$叫做\highlight{red}{\index{JFBL@积分变量}积分变量},$a$叫做\highlight{red}{\index{JFXX@积分下限}积分下限},$b$叫做\highlight{red}{\index{JFSX@积分上限}积分上限},$[a,b]$叫做\highlight{red}{\index{JFQJ@积分区间}积分区间}。\\
\kg 由于积分的定义与极限有关,故我们也可以用$\varepsilon-\delta$语言来表述定积分的定义。
\\ \kg 设有常数$I$,如果对于任意给定的正数$\varepsilon$,总存在一个正数$\delta$,使得对于区间$[a,b]$的任何分法,不论$\xi_i$在$[x_{i-1},x_i]$中怎样选取,只要$\lambda=\max\{\Delta x_i,1\leq i\leq n\}<\delta$,总有
\begin{equation}
	\bigg|\sum_{i=1}^{n}f(\xi_i)\Delta x_i-I\bigg|<\varepsilon\sj
\end{equation}
\noindent 成立,那么我们称$I$函数$f(x)$在区间$[a,b]$上的定积分,记作$\di\int_{a}^{b}f(x)\d x$.
\warn[\kg 定积分的值只与被积函数和被积区间有关,而与积分变量的记法无关。例如
\begin{equation}
	\int_{a}^{b}f(x)\d x=\int_{a}^{b}f(t)\d t=\int_{a}^{b}f(u)\d u\sj
	\end{equation}
\kg $\di\sum_{i=1}^{n}f(\xi_i)\Delta x_i$通常称为$f(x)$的积分和。如果$f(x)$在{$[a,b]$}上的定积分存在,那么就称$f(x)$在{$[a,b]$}上可积。]